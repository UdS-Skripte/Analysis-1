\documentclass{scrreprt}

%\usepackage[utf8]{inputenc}
\usepackage{enumerate}
\usepackage{ngerman}
%\usepackage[german]{babel}

\usepackage{amsmath}
\usepackage{amsfonts}
\usepackage{amssymb}
\usepackage{amsthm}
\usepackage{mathtools}

\usepackage{pgf,tikz}
\usepackage{mathrsfs}
\usetikzlibrary{arrows} 

\usepackage[hidelinks]{hyperref}

\makeatletter
\def\moverlay{\mathpalette\mov@rlay}
\def\mov@rlay#1#2{\leavevmode\vtop{%
   \baselineskip\z@skip \lineskiplimit-\maxdimen
   \ialign{\hfil$\m@th#1##$\hfil\cr#2\crcr}}}
\newcommand{\charfusion}[3][\mathord]{
    #1{\ifx#1\mathop\vphantom{#2}\fi
        \mathpalette\mov@rlay{#2\cr#3}
      }
    \ifx#1\mathop\expandafter\displaylimits\fi}
\renewcommand*\env@matrix[1][*\c@MaxMatrixCols c]{%
  \hskip -\arraycolsep
  \let\@ifnextchar\new@ifnextchar
  \array{#1}}
\makeatother


\usepackage{coloremoji}
\usepackage{extarrows}

\newcommand{\NN}{\mathbb{N}}
\newcommand{\ZZ}{\mathbb{Z}}
\newcommand{\QQ}{\mathbb{Q}}
\newcommand{\RR}{\mathbb{R}}
\newcommand{\CC}{\mathbb{C}}

\newcommand{\round}[1]{\left(#1\right)}
\newcommand{\brackets}[1]{\left[#1\right]}
\newcommand{\curly}[1]{\left\lbrace#1\right\rbrace}
\newcommand{\abs}[1]{\left\vert#1\right\vert}
\newcommand{\euler}{\mathrm{e}}

\newcommand{\cupdot}{\charfusion[\mathbin]{\cup}{\cdot}}
\newcommand{\bigcupdot}{\charfusion[\mathop]{\bigcup}{\cdot}}


\begin{document}

	\title{Analysis 1}
 	%\author{Dominic Zimmer, Alexander Yongsap\\in Kollaboration mit Jesko Dujmovic und Pascal Weber}
    \author{Jesko Dujmovic, Pascal Weber\\in Kollaboration mit Dominic Zimmer und Alexander Yongsap}
 	\subtitle{Übungsblatt 11, Abgabe}
 	\publishers{Übungsgruppe: Rami Ahmad}
  	\maketitle 

	%see http://www.namsu.de/Extra/strukturen/Inhaltsverzeichnis.html

\section*{Aufgabe 1}
Bestimmen sie die Ableitungen folgender Funktionen:
\begin{align*}
    f_1 \colon \round{0,\infty} \to \RR, x \mapsto x^a ~~~ a \in \RR\\
    f_2 \colon \RR \to \RR, x \mapsto b^x~~~ b >0\\
    f_3 \colon \round{0,\infty} \to \RR, x \mapsto x^x
\end{align*}
\begin{enumerate}[(a)]
\item
    \begin{align*}
        \round{f_1\round{x}}' & = \round{x^a}'\\
        & = \round{\euler^{a \ln\round{x}}}'\\
        & = \euler^{a \ln\round{x}} \cdot \round{a \ln\round{x}}' ~~~ \text{(Kettenregel)}\\
        & = \euler^{a \ln\round{x}} \cdot \frac{a}{x}\\
        & = a \cdot x^{a-1}
    \end{align*}
    \begin{align*}
        \round{f_2\round{x}}' & = \round{b^x}'\\
        & = \round{\euler^{x \ln\round{b}}}'\\
        & = \euler^{x \ln\round{b}} \cdot \round{x \ln\round{b}}' ~~~ \text{(Kettenregel)}\\
        & = \euler^{x \ln\round{b}} \cdot \ln\round{b}\\
        & = b^x \cdot \ln\round{b}
    \end{align*}
    \begin{align*}
        \round{f_3\round{x}}' & = \round{x^x}'\\
        & = \round{\euler^{x \ln\round{x}}}'\\
        & = \euler^{x \ln\round{x}} \cdot \round{x \ln\round{x}}' ~~~ \text{(Kettenregel)}\\
        & = \euler^{x \ln\round{x}} \cdot \round{\ln\round{x} + 1}\\
        & = x^x \cdot \round{\ln\round{x} + 1}
    \end{align*}
\item
    \begin{enumerate}[(i)]
    \item
        Wir überprüfen die Stetigkeit mithilfe des Folgenkriteriums. Sei dazu $(x_n)_{n \in \NN}$ eine beliebige Nullfolge
        \begin{align*}
            \lim\limits_{n \to \infty} f(x_n) & = \lim\limits_{n \to \infty} \round{x_n}^{\round{x_n}}\\
            & = \lim\limits_{n \to \infty} \euler^{x_n \cdot \ln\round{x_n}}\\
            & = \euler^0 ~~~ \text{(Wachstum von $\ln$)}\\
            & = 1\\
            & = f(0)
        \end{align*}
        Wir widerlegen die Differenzierbarkeit mit dem Differenzialquotienten und der in (a) ausgerechneten Ableitung.
        \begin{align*}
            \lim\limits_{x \to 0} \frac{f\round{x} - f\round{0}}{x-0} & = \lim\limits_{x \to 0} \frac{f\round{x} - 1}{x}\\
            & \overset{🏥}{=} \lim\limits_{x \to 0} \frac{f'\round{x}}{1} ~~~\text{(Da Zähler und Nenner gegen 0 streben)}\\
            & = \lim\limits_{x \to 0} x^x \cdot \round{\ln\round{x} + 1}\\
            & = -\infty
        \end{align*}
    \item
        Wir bestimmen die Monotonieintervalle durch die Signa der ersten Ableitung von $f$.
        \begin{align*}
            f'(x) = 0 &\Leftrightarrow x^x \cdot \round{\ln(x) + 1} = 0\\
            & \Leftrightarrow x = \euler^{-1} ~~~ \text{(Da $x^x > 0$)}\\
        \end{align*}
        Wir verifizieren noch zusätzlich, dass $f''(\euler^{-1})>0$, und somit ein lokales Minimum vorliegt. Da $f(0)= 1 > 0.69 \approx f(\euler^{-1}$ existieren keine Randextrema (da $\lim\limits_{x \to \infty} x^x = \infty$).
        \begin{align*}
            f''(x) & = x^x \cdot \round{\frac{1}{x} + \round{\ln\round{x} + 1}^2}\\
            f''(\euler^{-1}) & = \underbrace{ \underbrace{\round{\euler^{-1}}^{\round{\euler^{-1}}}}_{>0} \cdot \round{\underbrace{\euler}_{>0} + \underbrace{\round{\ln\round{\frac{1}{\euler}} + 1}^2}_{\geq0}}}_{>0}\\
        \end{align*}
        Mit $f(0) =1$ folgt nun, dass $f$ auf $[0,\frac{1}{\euler}]$ monoton fällt und auf $[\frac{1}{\euler},\infty)$.
    \item
        Da $f(\frac{1}{\euler})$ das globale Minimum ist und der Grenzwert $\lim\limits_{x \to \infty} x^x = \infty$ ist, folgt nach dem Zwischenwertsatz das Bild $f([0,\infty)) = [f\round{\frac{1}{\euler}}, \infty)$.
    \end{enumerate}
\end{enumerate}

\pagebreak
\section*{Aufgabe 2}
Seien $a,b >0$ relle Zahlen. Die Folgenden Grenzwerte gelten zu berechnen:
\begin{enumerate}[(a)]
\item
    \begin{align*}
        \lim\limits_{x \to a} \frac{x^5 - a^5}{x^3 - a^3} \xlongequal{🏥^0} \lim\limits_{x \to a} \frac{5 x^4}{3 x^2} = \frac{5}{3} a^2
    \end{align*}
\item
    \begin{align*}
        \lim\limits_{n \to \infty} \frac{a^{\frac{1}{n}} -1}{b^{\frac{1}{n}} -1} \xlongequal{🏥^0} \lim\limits_{n \to \infty} \frac{a^{\frac{1}{n}} \cdot \ln\round{a}}{b^{\frac{1}{n}} \cdot \ln\round{b}} \xlongequal{\text{linear}} \frac{\ln\round{a}}{\ln\round{b}} \cdot \lim\limits_{n \to \infty} \frac{\sqrt[n]{a}}{\sqrt[n]{b}} = \frac{\ln\round{a}}{\ln\round{b}}
    \end{align*}
\item
    \begin{align*}
        \lim\limits_{x \to 0} \frac{\euler^{ax} - \euler^{-ax}}{\sin\round{bx}} \xlongequal{🏥^0} \lim\limits_{x \to 0} \frac{a \cdot \euler^{ax} - (-a) \cdot \euler^{-ax}}{b\cos\round{bx}} = \lim\limits_{x \to 0} \frac{a \cdot \round{\euler^{ax} + \euler^{-ax}}}{b \cos\round{bx}}= \frac{2 a}{b}
    \end{align*}
\item
    \begin{align*}
        &\lim\limits_{x \to 0} \round{\frac{1}{\sin\round{x}} - \frac{1}{x}} = \lim\limits_{x \to 0} \round{\frac{x - \sin\round{x}}{x \sin\round{x}}} \xlongequal{🏥^0} \lim\limits_{x \to 0} \round{\frac{1-\cos\round{x}}{\sin\round{x} + x \cos\round{x}}} \\
        \xlongequal{🏥^0}& \lim\limits_{x \to 0} \round{\frac{\sin\round{x}}{\cos\round{x} + \cos\round{x} - x \sin\round{x}}} = \frac{0}{1+1+0} = 0
    \end{align*}
\end{enumerate}
Wobei wir "`$🏥^0$"' als Kurzschreibweise verwenden für "`gerechtfertigte Anwendung des Satzes von l'Hospital, da Nenner und Zähler des vorangegangenen Bruches im Grenzübergang gegen Null streben"'.

\pagebreak
\section*{Aufgabe 3}
Sei $f \colon \RR \to \RR$ eine überall differenzierbare Funktion mit den Grenzwerten
\begin{align*}
    \lim\limits_{x \to \infty} f\round{x} = \lim\limits_{x \to -\infty} f\round{x} = \infty
\end{align*}
\begin{enumerate}[(a)]
\item
    Zu zeigen ist, dass ein $x_0 \in \RR$ existiert, mit $f'(x_0) = 0$. Aufgrund des Grenzverhaltens von $f$, existieren $a < b \in \RR$ mit $f\round{a} = f\round{b}$ (entweder ist die Funktion konstant oder weist einen Monotoniewechsel auf). Der Mittelwertsatz liefert uns die Existenz eines $x_0 \in \RR$ folgender Gestalt:
    \begin{align*}
        f'(x_0) = \frac{f(a) - f(b)}{a-b} = \frac{0}{a-b} = 0
    \end{align*}
\item
    Allgemein existiert aber nicht zu jedem $a \in \RR$ ein $x_0 \in \RR$, so dass $f'(x_0) = a$ ist. Gegenbeispiel:
    \begin{align*}
        g \colon \RR \to \RR, x \mapsto \begin{cases} 
                                            \abs{x}, & \abs{x} > \frac{1}{2}\\
                                            \frac{1}{2}x^2,& \abs{x} \leq \frac{1}{2}
                                        \end{cases}
        ~~~ \text{mit} ~~~
        g' \colon \RR\setminus\curly{-\frac{1}{2}, \frac{1}{2}} \to \RR, x \mapsto \begin{cases} 
                                            \frac{\abs{x}}{x}, & \abs{x} > \frac{1}{2}\\
                                            x ,& \abs{x} < \frac{1}{2}
                                        \end{cases}
    \end{align*}
    Wobei wir die Betragsfunktion
    \begin{align*}
        \abs{\ \cdot \ }\colon \RR \to \RR, x \mapsto \begin{cases} 
                                            x, & x \geq 0\\
                                            -x,& x<0
                                        \end{cases},\\
    \end{align*}
    welche auf $\RR\setminus\curly{0}$ differenzierbar ist, verwenden.\\
    Das verlangte Grenzverhalten wird von $g$ erfüllt. Auch die Stetigkeit (auf ganz $\RR$) und die Differenzierbarkeit auf $\RR\setminus \curly{-\frac{1}{2}, \frac{1}{2}}$ lässt sich leicht nachprüfen (bzw. folgt als Verknüpfung differenzierbarer Funktionen trivial). Bleibt die Differenzierbarkeit in $\frac{1}{2}$ und $-\frac{1}{2}$ zu zeigen. Wir werden diese nur für $\frac{1}{2}$ zeigen, da der andere Fall ganz analog verläuft.
    \begin{align*}
        \lim\limits_{x \nearrow \frac{1}{2}} g'(x) = \lim\limits_{x \nearrow \frac{1}{2}} x = 1 = \lim\limits_{x \searrow \frac{1}{2}} \frac{\mathrm{\abs{x}}}{x} = \lim\limits_{x \searrow \frac{1}{2}} g'(x)
    \end{align*}
    Somit lässt sich $g'$ auf $\RR$ differenzierbar fortsetzen durch 
    \begin{align*}
        g'(\frac{1}{2}) = 1 ~~~ g'(-\frac{1}{2}) = -1
    \end{align*}
\end{enumerate}

\pagebreak
\section*{Aufgabe 5}
Sei $f \colon [a, b] \to \RR$ eine $C^1$-Funktion auf $[a,b]$ mit $-\infty < a < b < \infty$. Zu zeigen ist, dass $f$ Lipschitz-stetig ist mit Lipschitz-Konstante $L = \Vert f' \Vert_{[a,b]}$.\\
Da $f$ auf $[a,b]$ $C^1$ ist, können wir den Mittelwertsatz der Differentialrechnung anwenden: Es gibt also ein $\xi \in (a,b)$ mit:
\begin{align*}
    &f'(\xi) = \frac{f(b) - f(a)}{b-a}
\end{align*}
Schätzen wir nun gegen die Beträge und die Supremumsnorm ab, erhalten wir:
\begin{align*}
    & \abs{f(b) - f(a)} \\
    = &\abs{f'(\xi)} \cdot \abs{b-a} \\
    \leq & \Vert f' \Vert \cdot \abs{b-a}
\end{align*}
Also Lipschitz-stetig mit der gewünschten Lipschitz-Konstante.
\end{document}