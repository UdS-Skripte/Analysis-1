\documentclass{scrreprt}

\usepackage[utf8]{inputenc}
\usepackage{enumerate}
\usepackage{ngerman}
%\usepackage[german]{babel}

\usepackage{amsmath}
\usepackage{amsfonts}
\usepackage{amssymb}
\usepackage{amsthm}
\usepackage{mathtools}

\usepackage{pgf,tikz}
\usepackage{mathrsfs}
\usetikzlibrary{arrows} 

\usepackage[hidelinks]{hyperref}

\makeatletter
\def\moverlay{\mathpalette\mov@rlay}
\def\mov@rlay#1#2{\leavevmode\vtop{%
   \baselineskip\z@skip \lineskiplimit-\maxdimen
   \ialign{\hfil$\m@th#1##$\hfil\cr#2\crcr}}}
\newcommand{\charfusion}[3][\mathord]{
    #1{\ifx#1\mathop\vphantom{#2}\fi
        \mathpalette\mov@rlay{#2\cr#3}
      }
    \ifx#1\mathop\expandafter\displaylimits\fi}
\renewcommand*\env@matrix[1][*\c@MaxMatrixCols c]{%
  \hskip -\arraycolsep
  \let\@ifnextchar\new@ifnextchar
  \array{#1}}
\makeatother

\newcommand{\NN}{\mathbb{N}}
\newcommand{\ZZ}{\mathbb{Z}}
\newcommand{\QQ}{\mathbb{Q}}
\newcommand{\RR}{\mathbb{R}}
\newcommand{\CC}{\mathbb{C}}

\newcommand{\round}[1]{\left(#1\right)}
\newcommand{\brackets}[1]{\left[#1\right]}
\newcommand{\curly}[1]{\left\lbrace#1\right\rbrace}
\newcommand{\abs}[1]{\left\vert#1\right\vert}
\newcommand{\euler}{\mathrm{e}}

\newcommand{\round}[1]{\left\(#1\right\)}

\newcommand{\cupdot}{\charfusion[\mathbin]{\cup}{\cdot}}
\newcommand{\bigcupdot}{\charfusion[\mathop]{\bigcup}{\cdot}}


\begin{document}

	\title{Analysis 1}
 	\author{Dominic Zimmer, Alexander Yongsap\\in Kollaboration mit Jesko Dujmovic und Pascal Weber}
 	\subtitle{Übungsblatt 11, Abgabe}
 	\publishers{Übungsgruppe: Rami Ahmad}
  	\maketitle 

	%see http://www.namsu.de/Extra/strukturen/Inhaltsverzeichnis.html

\section*{Aufgabe 1}
Bestimmen sie die Ableitungen folgender Funktionen:
\begin{align*}
    f_1 \colon \round{0,\infty} \to \RR, x \mapsto x^a ~~~ a \in \RR\\
    f_2 \colon \RR \to \RR, x \mapsto b^x~~~ b >0\\
    f_3 \colon \round{0,\infty} \to \RR, x \mapsto x^x
\end{align*}
\begin{enumerate}[(a)]
\item
    \begin{align*}
        \round{f_1\round{x}}' & = \round{x^a}'\\
        & = \round{\euler^{a \ln\round{x}}}'\\
        & = \euler^{a \ln\round{x}} \cdot \round{a \ln\round{x}}' ~~~ \text{(Kettenregel)}\\
        & = \euler^{a \ln\round{x}} \cdot \frac{a}{x}\\
        & = a \cdot x^{a-1}
    \end{align*}
    \begin{align*}
        \round{f_2\round{x}}' & = \round{b^x}'\\
        & = \round{\euler^{x \ln\round{b}}}'\\
        & = \euler^{x \ln\round{b}} \cdot \round{x \ln\round{b}}' ~~~ \text{(Kettenregel)}\\
        & = \euler^{x \ln\round{b}} \cdot \ln\round{b}\\
        & = b^x \cdot \ln\round{b}
    \end{align*}
    \begin{align*}
        \round{f_3\round{x}}' & = \round{x^x}'\\
        & = \round{\euler^{x \ln\round{x}}}'\\
        & = \euler^{x \ln\round{x}} \cdot \round{x \ln\round{x}}' ~~~ \text{(Kettenregel)}\\
        & = \euler^{x \ln\round{x}} \cdot \round{\ln\round{x} + 1}\\
        & = x^x \cdot \round{\ln\round{x} + 1}
    \end{align*}
\item
    \begin{enumerate}[(i)]
    \item
        Wir überprüfen die Stetigkeit mithilfe des Folgenkriteriums. Sei dazu $(x_n)_{n \in \NN}$ eine beliebige Nullfolge
        \begin{align*}
            \lim\limits_{n \to \infty} f(x_n) & = \lim\limits_{n \to \infty} \round{x_n}^{\round{x_n}}\\
            & = \lim\limits_{n \to \infty} \euler^{x_n \cdot \ln\round{x_n}}\\
            & = \euler^0 ~~~ \text{(Wachstum von $\ln$)}\\
            & = 1\\
            & = f(0)
        \end{align*}
        Wir widerlegen die Differenzierbarkeit mit dem Differenzialquotienten und der in (a) ausgerechneten Ableitung.
        \begin{align*}
            \lim\limits_{x \to 0} \frac{f\round{x} - f\round{0}}{x-0} & = \lim\limits_{x \to 0} \frac{f\round{x} - 1}{x}\\
            & = \lim\limits_{x \to 0} \frac{f'\round{x}}{1} ~~~ \text{(del ospedale)}\\
            & = \lim\limits_{x \to 0} x^x \cdot \round{\ln\round{x} + 1}\\
            & = -\infty
        \end{align*}
    \item
        Wir bestimmen die Monotonieintervalle durch die Signa der ersten Ableitung von $f$.
        \begin{align*}
            f'(x) = 0 &\Leftrightarrow x^x \cdot \round{\ln(x) + 1} = 0\\
            & \Leftrightarrow x = \euler^{-1} ~~~ \text{(Da $x^x > 0$)}\\
        \end{align*}
        Wir verifizieren noch zusätzlich, dass $f''(\euler^{-1})>0$, und somit ein lokales Minimum vorliegt. Da $f(0)= 1 > 0.69 \approx f(\euler^{-1}$ existieren keine Randextrema (da $\lim\limits_{x \to \infty} x^x = \infty$).
        \begin{align*}
            f''(x) & = x^x \cdot \round{\frac{1}{x} + \round{\ln\round{x} + 1}^2}\\
            f''(\euler^{-1}) & = \underbrace{ \underbrace{\round{\euler^{-1}}^{\round{\euler^{-1}}}}_{>0} \cdot \round{\underbrace{\euler}_{>0} + \underbrace{\round{\ln\round{\frac{1}{\euler}} + 1}^2}_{>0}}}_{>0}\\
        \end{align*}
        Mit $f(0) =1$ folgt nun, dass $f$ auf $[0,\frac{1}{\euler}]$ monoton fällt und auf $[\frac{1}{\euler},\infty)$.
    \item
        Da $f(\frac{1}{\euler})$ das globale Minimum ist und der Grenzwert $\lim\limits_{x \to \infty} x^x = \infty$ ist, folgt nach dem Zwischenwertsatz das Bild $f([0,\infty)) = [f\round{\frac{1}{\euler}}, \infty)$.
    \end{enumerate}
\end{enumerate}


\section*{Aufgabe 3}


\end{document}