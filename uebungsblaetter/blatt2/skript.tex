\documentclass{scrreprt}

\usepackage{amsmath}
\usepackage{amssymb}
\usepackage[utf8]{inputenc}
\usepackage{enumerate}
\usepackage{ngerman}
\usepackage[utf8]{inputenc}
\usepackage[german]{babel}

\usepackage{amsmath}
\usepackage{amsfonts}
\usepackage{amssymb}
\usepackage{amsthm}

\usepackage[hidelinks]{hyperref}

\newenvironment{proof}{\emph{\\Beweis.}}{$\square$}

\makeatletter
\def\moverlay{\mathpalette\mov@rlay}
\def\mov@rlay#1#2{\leavevmode\vtop{%
   \baselineskip\z@skip \lineskiplimit-\maxdimen
   \ialign{\hfil$\m@th#1##$\hfil\cr#2\crcr}}}
\newcommand{\charfusion}[3][\mathord]{
    #1{\ifx#1\mathop\vphantom{#2}\fi
        \mathpalette\mov@rlay{#2\cr#3}
      }
    \ifx#1\mathop\expandafter\displaylimits\fi}
\makeatother

\newcommand{\NN}{\mathbb{N}}
\newcommand{\ZZ}{\mathbb{Z}}
\newcommand{\QQ}{\mathbb{Q}}
\newcommand{\RR}{\mathbb{R}}
\newcommand{\CC}{\mathbb{C}}
\newcommand{\cupdot}{\charfusion[\mathbin]{\cup}{\cdot}}
\newcommand{\bigcupdot}{\charfusion[\mathop]{\bigcup}{\cdot}}


\begin{document}

	\title{Analysis 1}
 	\author{Dominic Zimmer}
 	\subtitle{Übungsblatt 2, Abgabe}
 	%\date{18. Mai 1999}
  	\maketitle 

	%see http://www.namsu.de/Extra/strukturen/Inhaltsverzeichnis.html

	\section*{Aufgabe 1}
	\begin{samepage}
	\begin{enumerate}[a)]
		\item 
			Zu zeigen: $\forall n \in \NN_0$ mit $x \in \RR \setminus \lbrace 1 \rbrace$:
			\[
				\sum_{k=0}^n x^k = \frac{1 - x^{n + 1}}{1-x}
			\]
			\begin{proof}
				Beweis durch Vollständige Induktion über $n$. Induktionsanfang $n=0$:
				\[
					\sum\limits_{k=0}^0 x^k = 1 = \frac{1-x^{1}}{1-x^{1}} = \frac{1 - x^{n + 1}}{1-x}
				\]
				Induktionsschritt $n \leadsto n+1$.\\
				Sei die Behauptung bereits bewiesen für $n \in \NN$.
				\begin{align*}
					\sum_{k=0}^{n+1} x^k = & \sum_{k=0}^n x^k + x^{n+1}\\
					\overset{IV}{=} &\frac{1 - x^{n + 1}}{1-x} + x^{n+1})\\
					= & \frac{1 - x^{n + 1}}{1-x} + \frac{x^{n+1} \cdot (1-x)}{1-x}\\
					= & \frac{1 - x^{n + 1} + (x^{n+1} - x^{n+2})}{1-x}\\
					= & \frac{1 - x^{(n+1) + 1}}{1-x}
				\end{align*}
				Nach dem Prinzip der Vollständigen Induktion folgt die Behauptung.
			\end{proof}
		\item 
			Zu zeigen: $\forall n \in \NN_{>0}:$
			\[
				7^{n+1} + 8^{2n-1}
			\] 
			ist durch 57 teilbar.
			\begin{proof}
				Beweis durch Vollständige Induktion über $n$. Induktionsanfang $n=1$:
				\[
					7^{n+1} + 8^{2n-1} = 7^{2} + 8 = \underbrace{57}_{\text{durch 57 teilbar}}
				\]
				Induktionsschritt $n \leadsto n+1$.\\
				Sei die Behauptung bereits bewiesen für $n \in \NN$.
				\begin{align*}
					7^{(n+1) +1} + 8^{2(n+1)-1} = & 7^{n+2} + 8^{2n+1}\\
					= &7 \cdot 7^{n+1} + 8^2 \cdot 8^{2n-1}\\
					= &7 \cdot 7^{n+1} + (57+7) \cdot 8^{2n-1}\\
					= &\underbrace{7 \cdot (\underbrace{7^{n+1} + 8^{2n -1}}_{\text{durch 57 teilbar (IV)}}) + \underbrace{57 \cdot (8^{2n-1})}_{\text{57 als Faktor}}}_{\text{$\Rightarrow $durch 57 teilbar}}
				\end{align*}
				Nach dem Prinzip der Vollständigen Induktion folgt die Behauptung.
			\end{proof}
	\end{enumerate}
	\end{samepage}

	\section*{Aufgabe 2}
	\begin{enumerate}[a)]
		\item
			\begin{align*}
				A \setminus (A \setminus B) =& A \setminus \lbrace x \mid x \in A \land x \notin B\rbrace\\
				=& \lbrace x \mid x \in A \land x \notin \lbrace y \mid y \in A \land y \notin B \rbrace \rbrace\\
				=& \lbrace x \mid x \in A \land x \in \lbrace y \mid y \notin A \lor y \in B \rbrace \rbrace\\
				=& \lbrace x \mid x \in A \land x \in B\rbrace\\
				=& A \cap B
			\end{align*}
		\item
			\begin{align*}
				(A \cup B) \cap C = & \lbrace x \mid x \in A \lor x \in B \rbrace \cap \lbrace x \mid x \in C\rbrace\\
				= & \lbrace x \mid (x \in A \lor x \in B )\land x \in C\rbrace \\
				= & \lbrace x \mid (x \in A \land x \in C) \lor( x \in B \land x \in C)\rbrace \\
				= & (A \cap C) \cup (B \cap C)
			\end{align*}
		\item
			\begin{align*}
				(A \cap B) \cup C = & \lbrace x \mid x \in A \land x \in B \rbrace \cup \lbrace x \mid x \in C\rbrace\\
				= & \lbrace x \mid (x \in A \land x \in B )\lor x \in C\rbrace \\
				= & \lbrace x \mid (x \in A \lor x \in C) \land( x \in B \lor x \in C)\rbrace \\
				= & (A \cup C) \cap (B \cup C)
			\end{align*}
	\end{enumerate}

	\pagebreak
	\section*{Aufgabe 3}
	\begin{enumerate}[a)]
		\item
			Wir unterscheiden in $f_n \colon \RR \to \RR$ zwischen drei Fällen:
			\begin{itemize}
				\item
					$n = 0$:\\
					Wir betrachten nun die Konstante Funktion $f_0(x) = x^0 = 1$, welche
					\begin{itemize}
						\item
							\emph{nicht injektiv} ist, da $\forall x_1, x_2 \in \RR: f(x_1) = 1 = f(x_2)$.
						\item
							\emph{nicht surjektiv} ist, da $\nexists x \in \RR: f(x) = 2$.
						\item
							folglich \emph{nicht bijektiv} ist.
					\end{itemize}
				\item
					$n \text{ mod }  2 = 0$:\\
					Nun betrachten wir $f_n(x) = x^{2k}$ mit $k \in \NN$. $f_n$ ist somit
					\begin{itemize}
						\item
							\emph{nicht injektiv}, da $\forall x \in \RR: f(x) = x^{2k} = (-x)^{2k} = f(-x)$
						\item
							\emph{nicht surjektiv}, da $\nexists x \in \RR: f(x) = -1$
						\item
							folglich \emph{nicht bijektiv}.
					\end{itemize}
				\item
					$n \text{ mod }  2 = 1$:\\
					Nun betrachten wir $f_n(x) = x^{2k - 1}$ mit $k \in \NN$. $f_n$ ist somit
					\begin{itemize}
						\item
							\emph{injektiv}, da $f_n$ offensichtlich monoton wächst
						\item
							\emph{surjektiv}, da $\lim\limits_{x \to -\infty} = -\infty$ und $\lim\limits_{x \to \infty} = \infty$ nach dem Zwischenwertsatz.	
						\item
							folglich \emph{bijektiv}.
					\end{itemize}
			\end{itemize}
			\item
				$f: \lbrace 1,2, \dots, n\rbrace \to \lbrace 0, 1, 2, \dots , n^2 -1 \rbrace$ mit $f(x) = x^2 -1$ ist
				\begin{itemize}
						\item
							\emph{injektiv}, da $f|_{[1, n]}$ offensichtlich streng monoton steigt.
						\item
							\emph{nicht surjektiv}, da $\nexists x \in \lbrace 1, 2, \dots, n\rbrace: x^2 - 1 = 2$
						\item
							folglich \emph{nicht bijektiv}.
				\end{itemize}
			\item
				$f: \lbrace 0, 1, \dots, n \rbrace \to \lbrace 1, 2, \dots , \n+1 \rbrace$ mit $f(x) = x+1$ ist offensichtlich nur eine im Wertebereich geshiftete Identitätsabbildung, also
				\begin{itemize}
					\item
						\emph{bijektiv}, da $id_x$ nach Konstruktion bijektiv ist
					\item
						\emph{injektiv} und \emph{surjektiv} folglich.
				\end{itemize}
	\end{enumerate}

	\pagebreak
	\section*{Aufgabe 4}
	Seien $f \colon X \to Y$ und $g \colon Y \to Z$ Abbildungen.
	\begin{enumerate}[a)]
		\item
			\begin{enumerate}[(i)]
				\item
					Zu zeigen: \glqq$g \circ f \colon X \to Z$ injektiv\grqq \ impliziert \glqq$f$ injektiv\grqq.\\
					Da $g \circ f$ injektiv ist, lässt sich eindeutig von g(f(x)) auf f(x) schließen. Somit gilt
					\begin{align*}
						& g \circ f (x) = g \circ f (y) \Rightarrow x = y\\
						\Leftrightarrow & g(f(x)) = g(f(y)) \Rightarrow x = y\\
						\Rightarrow & f(x) = f(y) \Rightarrow x = y
					\end{align*}
					Also ist f auch injektiv.
				\item
					Zu zeigen: \glqq$g \circ f \colon X \to Z$ surjektiv\grqq \  impliziert \glqq$g$ surjektiv\grqq.\\
					Da $g \circ f$ surjektiv ist, gilt
					\begin{align*}
						&\forall x \in X: \exists z \in Z: g \circ f (x) = z\\
						&\forall x \in X: \exists z \in Z: g(f (x)) = z\\
						&\text{Wählt man nun $\hat{x} := f(x) \in Y$, erhält man}\\
						&\forall \hat{x} \in Y: \exists z \in Z: g(\hat{x}) = z
					\end{align*}
					Also ist g auch surjektiv.
			\end{enumerate}
		\item
			Da $id_X$ die Identitätsabbildung ist, ist sie bijektiv. Insbesondere ist $id_X$ also \emph{injektiv} und \emph{surjektiv}. Aus Aufgabenteil (a) folgt nun, dass $f$ injektiv und $g$ surjektiv sind.\\
			$f$ muss jedoch nicht \emph{surjektiv} und $g$ nicht \emph{injektiv} sein. Im folgenden Beispiel gilt $g \circ f = id_x$, dennoch ist $f$ nicht surjektiv und $g$ nicht injektiv.
			\begin{align*}
				f \colon \lbrace 1,2,3 \rbrace  \to  \lbrace 1,2,3,4,5,6\rbrace, f(x) = 2 \cdot x\\
				g \colon \lbrace 1,2,3,4,5,6 \rbrace   \to  \lbrace 1,2,3\rbrace, g(x) = \lceil \frac{x}{2}\rceil
			\end{align*}
	\end{enumerate}

	\pagebreak
	\section*{Aufgabe 5}
	\begin{enumerate}[a)]
		\item
			Zu zeigen: $|2^{N}| = 2^{|N|}$
			\begin{proof}
				Beweis durch vollständige Induktion über die Kardinalität einer Menge $N$. Induktionsanfang: $|N| = 0$:
				\[
					|2^{N}| = |2^{\emptyset}| = |\lbrace \emptyset \rbrace| = 1 = 2^{0} = 2^{|\emptyset|} 
				\]
				Induktionsschritt $n \leadsto n+1$.\\
				Sei nun $|N| = n$ und die Behauptung bereits gezeigt für $N$.\\
				Betrachten wir nun $M := N \cupdot \lbrace x \rbrace$. Offensichtlich gilt:
				\[
					2^M = 2^{N \cupdot \lbrace x \rbrace} = 2^N \cupdot \lbrace x \mid x = A \cupdot \lbrace x \rbrace, A \in 2^N  \rbrace
				\]
				Wir zerlegen $2^M$ also in zwei disjunkte Mengen: die Mengen, die aus $2^N$ stammen und die Mengen, die $x$ enthalten. Man sieht leicht, dass beide Mengen gleichmächtig sind.
				Somit können wir für die Kardinalitäten folgern:
				\[
					|2^M| = | 2^N \cupdot \lbrace x \mid x = A \cupdot \lbrace x \rbrace, A \in 2^N  \rbrace| \overset{\text{IV}}{=} 2^{|N|} + 2 ^{|N|} = 2^{|N|+1}
				\]
				Nach dem Prinzip der Vollständigen Induktion folgt die Behauptung.
			\end{proof}
		\item
			Zu zeigen: Die Potenzmenge der Natürlichen Zahlen ist überabzählbar.
			\begin{proof}
				Beweis durch Widerspruch. Wir nehmen an, dass die Potenzmenge der Natürlichen abzählbar ist. Somit könnten wir die Potenzmenge der natürlichen Zahlen auflisten.
				Wir kodieren nun binär Teilmengen von $\NN$, indem wir entscheiden, ob eine Zahl $x$ in der Menge enthalten ist, oder nicht.\\
				\begin{center}
					\begin{tabular}{c | c | c | c | c | c | c}
						$A_i \subseteq \NN$ & 1 & 2 & 3 & 4 & 5 &$\dots$\\
						\hline
						$A_1 = \lbrace 2, 3, \dots \rbrace$ & 0 & 1 & 1 & 0 & 0 &\\
						$A_2 = \lbrace 1, 4,5,  \dots \rbrace$ & 1 & 0 & 0 & 1 & 1 &\\
						$A_3 = \lbrace 3, 4,  \dots \rbrace$ & 0 & 0 & 1 & 1 & 0 &\\
					\end{tabular}
				\end{center}
				Konstruieren wir nun eine Menge $B \subseteq 2^{\NN}$
				\[
					B = \bigcupdot\limits_{A_i \subseteq \NN} \overline{A_i} \cap i
				\]
				sehen wir, dass sich $B$ (nach Konstruktion) von jedem $A_i$ in der Liste gerade in der Zahl $i$ unterscheided. Also ist B nicht in der Liste enthalten.
			\end{proof}
		\end{enumerate}

\end{document}