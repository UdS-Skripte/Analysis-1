\documentclass{scrreprt}[twoside=true]

\usepackage{amsmath}
\usepackage{amssymb}
\usepackage[utf8]{inputenc}

\begin{document}

	\tableofcontents

	%see http://www.namsu.de/Extra/strukturen/Inhaltsverzeichnis.html

	\chapter{Beweisprinzipien}

	\section{Aussagenlogik}

	

		Die Aussagenlogik befasst sich mit \emph{Aussagen}, welche \emph{(w)ahr} oder \emph{(f)alsch} sein können. Aus den Operatoren 
		\begin{itemize}
			\item
				Negation:
				\begin{equation*}
					\neg a =
						\begin{cases}
							w & \text{falls $a \equiv f$}.\\
							f & \text{falls $a \equiv w$}.
						\end{cases}
				\end{equation*}
			\item
				Konjunktion:
				\begin{equation*}
					a \lor b =
						\begin{cases}
							w & \text{falls $a \equiv w$ oder $b \equiv w$ (oder beide)}.\\
							f & \text{sonst}.
						\end{cases}
				\end{equation*}
			\item
				Disjunktion:
				\begin{equation*}
					a \land b =
						\begin{cases}
							w & \text{falls $a \equiv w$ und $b \equiv w$}.\\
							f & \text{sonst}.
						\end{cases}
				\end{equation*}
			\item
				Implikation:
				\begin{equation*}
					a \rightarrow b =
						\begin{cases}
							f & \text{falls $a \equiv w$ und $b \equiv f$}.\\
							w & \text{sonst}.
						\end{cases}
				\end{equation*}
			\item
				Äquivalenz:
				\begin{equation*}
					a \leftrightarrow b =
						\begin{cases}
							w & \text{falls $a \equiv b$}.\\
							f & \text{sonst}.
						\end{cases}
				\end{equation*}
		\end{itemize}
		lassen sich aus bereits bestehenden aussagelogischen Ausdrücken Weitere bilden. Auch lassen sich einfach aus den Definitionen Gesetzmäßigkeiten ableiten.

		\section{Axiome}
		\emph{Axiome} sind grundliegende Aussagen, die nicht weiter zurückgeführt werden (können). Wir \emph{beweisen}, indem wir \emph{Aussagen} auf Axiome zurückführen.

		\section{Direkter Beweis}
		Ein \emph{Direkter Beweis} wird geführt, indem man eine Aussage $A$ annimmt und ausgehend von dieser eine weitere Aussage $B$ \emph{beweist}.

		\subsection{Beispiel}
		Wir wollen zeigen, dass folgende Aussage korrekt ist:
		\begin{center}
			Das Quadrat einer geraden Zahl ist wiederum gerade.
		\end{center}
		\begin{proof}
			Sei $a \in \mathbb{N}$ eine gerade Zahl, welche sich also auch als $a = 2 \cdot k$ darstellen lässt. Betrachten wir nun das Quadrat von $a$, so gilt:
			\begin{equation*}
				a^2 = (2 \cdot k)^2 = 2^2 \cdot k^2 = 4 k^2 = 2 \cdot (2 k^2)
			\end{equation*}
			Somit hat also $a^2 = 2 \cdot (2 k^2)$ eine Zwei als Teiler und ist somit gerade.\square
		\end{proof}

		\section{Indirekter Beweis durch Kontraposition}
		Statt die Implikation $A \rightarrow B$ zu beweisen, können wir auch $\neg B \rightarrow \neg A$ beweisen. Wir nehmen also an, dass das zu zeigende nicht gilt und folgern daraus, dass unsere Annahme nicht gilt.

		\section{Widerspruch}
		Wir können eine Aussage $A$ auch beweisen, indem wir $\neg A$ annehmen und daraus einen Widerspruch folgern.

		\section{Prinzip der vollständigen Induktion}
		Ist $A(n)$ eine Aussage mit $n \in \mathbb{N}$, so können wir diese Gültigkeit dieser Aussage für alle $n > n_0$ zeigen, indem wir
		\begin{itemize}
			\item
				Die Gültigkeit der Aussage $A(n_0)$ zeigen und
			\item
				Aus der Annahme, dass die Aussage $A(n)$ für ein festes $n \in \mathbb{n}$ bereits gilt, darauf schließen, dass auch $A(n + 1)$ gilt.
		\end{itemize}

		\section{Summennotation}
		Seien $a_i$ ($i \in \mathbb{N}$) eine Familie von Zahlen. Wir führen folgende Kurzschreibweise ein:
		\begin{equation*}
			\sum\limits_{k=m}^n a_i = a_m + \dots + a_n
		\end{equation*}

		\section{Satz: Gaußformel}
		\begin{equation*}
				\sum\limits_{k=1}^n k = \frac{n \cdot (n + 1)}{2}
		\end{equation*}
		\begin{proof}
			Der Beweis erfolg einfach durch Induktion oder alternativ durch geschicktes, zweifaches Summieren obiger Summe.  
		\end{proof}

		

\end{document}
