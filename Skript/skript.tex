\documentclass{scrreprt}

\usepackage{amsmath}
\usepackage{amssymb}
\usepackage[utf8]{inputenc}
\usepackage{enumerate}
\usepackage{ngerman}
\usepackage{fancyhdr}
\usepackage{mathtools}

\usepackage[hidelinks]{hyperref}

\newenvironment{proof}{\emph{\\Beweis.}}{}

\pagestyle{fancy}

\newcommand{\NN}{\mathbb{N}}
\newcommand{\ZZ}{\mathbb{Z}}
\newcommand{\QQ}{\mathbb{Q}}
\newcommand{\RR}{\mathbb{R}}
\newcommand{\CC}{\mathbb{C}}
\newcommand{\FF}{\mathbb{F}}
\newcommand{\lecturedate}[1]{\pagebreak\rfoot{Lecture on: #1}}

\begin{document}

    \title{Analysis 1 Skript}
     \author{Dominic Zimmer}
     %\date{18. Mai 1999}
      \maketitle 

    \tableofcontents

    %see http://www.namsu.de/Extra/strukturen/Inhaltsverzeichnis.html

    \lecturedate{21.10.2015}

    \chapter{Beweisprinzipien}

    \section{Logik}    
    Die Aussagenlogik befasst sich mit \emph{Aussagen}, welche \emph{(w)ahr} oder \emph{(f)alsch} sein können. Aus den Operatoren 
    \begin{itemize}
        \item
            Negation:
            \begin{equation*}
                \neg a =
                    \begin{cases}
                        w & \text{falls $a \equiv f$}.\\
                        f & \text{falls $a \equiv w$}.
                    \end{cases}
            \end{equation*}
        \item
            Konjunktion:
            \begin{equation*}
                a \lor b =
                    \begin{cases}
                        w & \text{falls $a \equiv w$ oder $b \equiv w$ (oder beide)}.\\
                        f & \text{sonst}.
                    \end{cases}
            \end{equation*}
        \item
            Disjunktion:
            \begin{equation*}
                a \land b =
                    \begin{cases}
                        w & \text{falls $a \equiv w$ und $b \equiv w$}.\\
                        f & \text{sonst}.
                    \end{cases}
            \end{equation*}
        \item
            Implikation:
            \begin{equation*}
                a \rightarrow b =
                    \begin{cases}
                        f & \text{falls $a \equiv w$ und $b \equiv f$}.\\
                        w & \text{sonst}.
                    \end{cases}
            \end{equation*}
        \item
            Äquivalenz:
            \begin{equation*}
                a \leftrightarrow b =
                    \begin{cases}
                        w & \text{falls $a \equiv b$}.\\
                        f & \text{sonst}.
                    \end{cases}
            \end{equation*}
    \end{itemize}
    lassen sich aus bereits bestehenden aussagelogischen Ausdrücken Weitere bilden. Auch lassen sich einfach aus den Definitionen Gesetzmäßigkeiten ableiten.

    \section{Axiome}
    \emph{Axiome} sind grundliegende Aussagen, die nicht weiter zurückgeführt werden (können). Wir \emph{beweisen}, indem wir \emph{Aussagen} auf Axiome zurückführen.

    \section{Direkter Beweis}
    Ein \emph{Direkter Beweis} wird geführt, indem man eine Aussage $A$ annimmt und ausgehend von dieser eine weitere Aussage $B$ \emph{beweist}.

    \subsection{Beispiel}
    Wir wollen zeigen, dass folgende Aussage korrekt ist:
    \begin{center}
        Das Quadrat einer geraden Zahl ist wiederum gerade.
    \end{center}
    \begin{proof}
        Sei $a \in \NN$ eine gerade Zahl, welche sich also auch als $a = 2 \cdot k$ darstellen lässt. Betrachten wir nun das Quadrat von $a$, so gilt:
        \begin{equation*}
            a^2 = (2 \cdot k)^2 = 2^2 \cdot k^2 = 4 k^2 = 2 \cdot (2 k^2)
        \end{equation*}
        Somit hat also $a^2 = 2 \cdot (2 k^2)$ eine Zwei als Teiler und ist somit gerade.$\square$
    \end{proof}

    \section{Kontraposition}
    Statt die Implikation $A \rightarrow B$ zu beweisen, können wir auch $\neg B \rightarrow \neg A$ beweisen. Wir nehmen also an, dass das zu zeigende nicht gilt und folgern daraus, dass unsere Annahme nicht gilt.

    \section{Widerspruch}
    Wir können eine Aussage $A$ auch beweisen, indem wir $\neg A$ annehmen und daraus einen Widerspruch folgern.

    \section{Induktion}
    Das Prinzip der \emph{vollständigen Induktion} besagt: \\
    Ist $A(n)$ eine Aussage mit $n \in \NN$, so können wir diese Gültigkeit dieser Aussage für alle $n > n_0$ zeigen, indem wir
    \begin{itemize}
        \item
            Die Gültigkeit der Aussage $A(n_0)$ zeigen und
        \item
            Aus der Annahme, dass die Aussage $A(n)$ für ein festes $n \in \NN$ bereits gilt, darauf schließen, dass auch $A(n + 1)$ gilt.
    \end{itemize}

    \section{Summennotation}
    Seien $a_i$ ($i \in \NN$) eine Familie von Zahlen. Wir führen folgende Kurzschreibweise ein:
    \begin{equation*}
        \sum\limits_{k=m}^n a_i = a_m + \dots + a_n
    \end{equation*}

    \section{Gaußformel}
    Für alle $n \in \NN$ gilt: 
    \begin{equation*}
            \sum\limits_{k=1}^n k = \frac{n \cdot (n + 1)}{2}
    \end{equation*}
    \begin{proof}
        Der Beweis erfolg einfach durch Induktion oder alternativ durch geschicktes, zweifaches Summieren obiger Summe.  
    \end{proof}

    \section{Fakultät}
        \subsection{Notation}
        Wir definieren
        \begin{equation*}
            n ! = 1 \cdot 2 \cdot 3 \cdot \dots \cdot n = \prod\limits_{k=1}^n k 
        \end{equation*}
        als die \emph{Fakultät} von $n \in \NN$.
        \subsection{Binomailkoeffizient}
        Wir verwenden die Fakultät zur Definition des \emph{Binomialkoeffizientens}, den wir als \emph{n über k} oder \emph{k aus n} lesen:
        \begin{equation*} 
            \binom{n}{k} = \frac{n!}{k! \cdot (n-k)!}
        \end{equation*}

    \section{Lemma: Binomialkoeffizient}
    Für alle $1 \leq k \leq n$ gilt:
    \begin{equation*}
        \binom{n}{k} = \binom{n-1}{k-1} + \binom{n-1}{k}
    \end{equation*}
    \begin{proof}
        Nachrechnen. Eine Intuition für die Korrektheit erhält man leicht durch das \emph{Pascal'sche Dreieck}.
    \end{proof}

    \section{Binomischer Lehrsatz}
    Für alle $x, y \in \RR$ und $n \in \NN$ gilt:
    \begin{equation*}
        (a+b)^n = \sum\limits_{k=0}^n \binom{n}{k} a^k \cdot b^{n-k}
    \end{equation*}
    \begin{proof}
        Der Beweis erfolgt leicht durch Induktion über n.
    \end{proof}

    \lecturedate{26.10.2015}

    \chapter{Mengen}

    \section{Mengen nach Cantor}
    Cantos naive Mengendefinition besagte:\\
    Eine \emph{Menge} ist eine Zusammenfassung wohlbestimmter und wohlunterscheidbarer Objekte unserer Anschauung oder unseres Denkens zu einem Ganzen.\\ \\
    Diese \emph{naive} Definition birgt einige Widersprüche; zum Beispiel erlaubt sie die Menge aller Mengen.    

    \subsection{Schreibweisen}
    Wir benutzen folgende Schreibweisen im Umgang mit Mengen:
    \begin{itemize}
        \item
            $M = \lbrace x_1, x_2, \dots , x_n\rbrace$: Die Menge mit den Elementen $x_1, x_2, \dots, x_n$ und \emph{Kardinalität} $\# M = |M| = n$
        \item
            $x \in M$ : $x$ ist Element der Menge $M$
        \item
            $N \subset M$: $N$ ist eine \emph{Teilmenge} der Menge $M$
    \end{itemize}

    \section{Mengenoperatoren}
    Außerdem definieren wir für Zwei Mengen $M$ und $N$
    \begin{enumerate}[i)]
        \item
            die Vereinigung von $M$ und $N$:\\
            \[
                M \cup N = \lbrace x \mid x \in M \lor x \in N \rbrace
            \]
        \item
            den Schnitt von $M$ und $N$.
            \[
                M \cap N = \lbrace x \mid x \in M \land x \in N \rbrace
            \]
        \item
            das Komplement von $M$ in $N$.
            \[
                M \setminus N = \lbrace x \mid x \in M \land x \notin N \rbrace
            \]
    \end{enumerate}

    \section{Wichtige Mengen}
        Einige wichtige Mengen sind:
        \begin{itemize}
            \item
                $\emptyset = \lbrace \rbrace$, die \emph{Leere Menge}, welche keine Elemente hat.
            \item
                $\NN = \lbrace 1, 2, 3, \dots \rbrace$, die \emph{Natürlichen Zahlen}
            \item
                $\ZZ = \lbrace 0, 1, -1, 2, -2, \dots \rbrace$, die \emph{Ganzen Zahlen}
            \item
                $\QQ = \lbrace \frac{p}{q} \mid p,q \in \ZZ, q \neq 0\rbrace$, die \emph{Rationalen Zahlen}
            \item
                $\RR$, die Menge der \emph{Reellen Zahlen}
            \item
                $\CC = \lbrace (a, b \cdot i) \mid a, b \in \RR \rbrace$
         \end{itemize}

    \section{Quantoren}
    \emph{Quantoren} sind Kurzschreibweisen für in der Mathematik häufig benutzte Flosskeln. $\exists$ nennt man \emph{Existensquantor} und $\forall$ \emph Allquantor. Sei nun $X$ eine Menge und $P(x)$ eine Aussage über $x$.
    \begin{itemize}
        \item
            $\forall x \in X : P(x)$ für \glqq Für alle $x \in X$ gilt die Aussage $P(x)$.\grqq
        \item
            $\exists x \in X : P(x)$ für \glqq Es gibt (zumindest) ein $x \in X$ für welches die Aussage $P(x)$ gilt.\grqq
    \end{itemize}

    \section{Verneinung von Quantoren}
    Ausdrücke, welche Quantoren enthalten, werden Verneint, indem man den jeweiligen Existens- oder Allquantor mit dem Anderen ersetzt, und den Ausdruck dahinter verneint. Zum Beispiel:
    \begin{align*}
         & \neg \forall x \in X: \exists y \in Y : P (x,y)\\
        =& \exists x \in X: \neg \exists y \in Y : P(x,y)\\
        =& \exists x \in X: \forall y \in Y: \neg P(x,y)
    \end{align*}

    \section{Vereinigung und Schnitt}
    Sei $I \subseteq \NN$ eine Indexmenge und $M_i$ eine Familie von Mengen. Wir notieren
    \begin{itemize}
        \item
            $\bigcup\limits_{i \in I} M_i = M_{i_1} \cup M_{i_2} \cup \dots = \lbrace x \mid \exists i \in I: x \in M_i\rbrace$
        \item
            $\bigcap\limits_{i \in I} M_i = M_{i_1} \cap M_{i_2} \cap \dots = \lbrace x \mid \forall i \in I: x \in M_i\rbrace$
    \end{itemize}

    \section{De Morgan}
    Sei $M_i$ eine Familie von Mengen, so ist
    \begin{itemize}
        \item
             $\overline{\bigcup\limits_{i \in I} M_i} = \bigcap\limits_{i \in I} \overline{M_i} $
        \item
             $\overline{\bigcap\limits_{i \in I} M_i} = \bigcup\limits_{i \in I} \overline{M_i} $
    \end{itemize}

    \section{Abbildungen}
    \subsection{Definition}
         Seien $X$ und $Y$ Mengen. Wir definieren eine \emph{Abbildung} oder auch \emph{Funktion}
         \begin{equation*}
             f: x \longrightarrow Y
         \end{equation*}
         als eine Vorschrift, die jedem $x \in X$ genau ein $y \in Y$ zurodnet. Wir nennen dabei $X$ den \emph{Definitionsbereich} und $Y$ den \emph{Wertebereich}.
         \subsection{Eigenschaften}
         Wir nennen eine Abbildung $f: X \longrightarrow Y$
         \begin{itemize}
             \item
                 \emph{injektiv}, wenn $\forall x_1, x_2 \in X: f(x_1) = f(x_2) \rightarrow x_1 = x_2$
             \item
                 \emph{surjektiv}, wenn $\forall y \in Y : \exists x \in X: f(x) = y$
             \item
                 \emph{bijektiv}, wenn f injektiv und surjektiv ist.
        \end{itemize}

    \lecturedate{28.10.2015}

    \section{Komposition}
    Seien $f: x \longrightarrow Y$ und $g: Y \longrightarrow Z$ Abbildungen. Wir definieren die \emph{Komposition} $g \circ f : X \longrightarrow Z$ definiert durch
    \begin{equation*}
        g \circ f := g(f(x)) \text{ für } x \in X.    
    \end{equation*}

    \section{Identitätsabbildung}
    Wir nennen $id_x : X \rightarrow X$ die \emph{Identitätsabbildung} auf $X$ mit
    \begin{equation*}
        id_X (x) = X, \forall x \in X
    \end{equation*}
    Sie fungier als das Neutrale Element der Komposition von Funktionen.

    \section{Umkehrabbildung}
    Sei $f : X \rightarrow Y$ eine Bijketion. Wir definieren die \emph{Umkehrabbildung} $f^{-1}$ von $f$ durch
    \begin{equation*}
        f^{-1} : Y \longrightarrow X, f^{-1}(y) = x \text{ mit } f(x) = y 
    \end{equation*} 
    Woraus offensichtlich folgt, dass $f \circ f^{-1} = id_X$

    \section{Kardinalitäten}
        \subsection{Definition}
        Zwei Mengen $N$ und $M$ sind \emph{gleichmächtig}, falls eine Bijektion $f: N \longrightarrow M$ existiert.
        \subsection{Abzählbar}
        Eine Menge $M$ heißt \emph{abzählbar}, falls sie entweder \emph{endlich} oder \emph{gleichmächtig wie $\NN$} ist.
        \subsection{Überabzählbar}
        Eine Menge $M$, die nicht abzählbar ist, nennen wir \emph{überabzählbar}.

    \section{Kardinalität von $\RR$}
    $\RR$ ist überabzählbar.
    \begin{proof}
        Offensichtlich genügt es zu zeigen, dass eine Teilmenge von $\RR$ überabzählbar ist, um zu zeigen, dass $\RR$ überabzählbar ist. Betrachten wir also das Intervall $[0,1]$. Wir wollen einen Widerspruchsbeweis führen. Nehmen wir also an, $\RR$ sei abzählbar. So könnten wir also alle Zahlen aus $\RR$ abzählen.\\
        \begin{center}
            \begin{tabular}{ll}
                $1$ & $0.a_1 a_2 a_3 a_4 a_5 \dots$\\
                $2$ & $0.b_1 b_2 b_3 b_4 b_5 \dots$\\
                $3$ & $0.c_1 c_2 c_3 c_4 c_5 \dots$\\
                $4$ & $0.d_1 d_2 d_3 d_4 d_5 \dots$\\
                $\vdots$ & $\ddots$  
            \end{tabular}
        \end{center}
        Konstruieren wir nun eine Zahl $z$, welche stehts in der $n-ten$ Nachkommastelle mit der $n-ten$ Zahl der Liste \emph{nicht} übereinstimmt. Also kann $z$ nicht die erste Zahl der Liste sein, da sie in der ersten Nachkommastelle nicht mit ihr übereinstimmt. Dies läuft darauf hinaus, dass $z$ mit jeder Zahl aus der Liste in der $n-ten$ Nachkommastelle \emph{nicht} übereinstimmt. Also ist $z$ nicht in der Liste. Somit ist $[0,1]$ nicht abzählbar und somit ist $\RR$ nicht abzählbar.
    \end{proof}
    
    \chapter{Folgen}
    \section{Folgen reeller Zahlen}
        \subsection{Definition}
        Eine Folge reeller Zahlen ist eine Abbildung $\NN \longrightarrow \RR , n \mapsto a_n $. Schreibweise: $ (a_n ) n \in \NN $ oder $ (a_1 , a_2 ,a_3 , \dots )$
        Folgen müssen nicht mit dem Index 1 beginnen, auch Folgen der Form $(a_n ) n \geq n_0 $ sind unmöglich.
        
        \subsection{Beispiele von Folgen}
        \begin{enumerate}[i)]
            \item $a_n =2 ,\forall n \in \NN$ also $(2,2,2,\dots )$
            \item $a_n = n ,\forall n \in \NN$ also $(1,2,3,\dots )$
            \item $a_n = \frac{1}{n},\forall n \in \NN$ also $(\frac{1}{1},\frac{1}{2},\frac{1}{3}, \dots )$
            \item $a_n =(-1)^n , \forall n \in \NN$ also $(-1,1,-1,1, \dots )$
        \end{enumerate}
    \section{Konvergenz}
        \subsection{Definition}
        Eine Folge$(a_n ) n \in \NN$ reeller Zahlen heißt konvergent mit Grenzwert $a \in \RR$(oder \glqq konvergent gegen $a$ \grqq ), falls $\forall \varepsilon > 0 :\exists N \in \NN :\forall n \geq N:|a-a_n | < \varepsilon$. Schreibweise: $\lim\limits_{n \rightarrow \infty}{a_n}$ \\
        $a_n \longrightarrow a, \lim n \longrightarrow \infty$
        \subsection{Bemerkung}
        Für $x,y \in \RR$ gilt die inverse Dreiecksgleichung
        \begin{equation*}
            ||x|-|y|| \leq |x-y|
        \end{equation*}

    \section{TODO: FEHLT!!}

    \lecturedate{02.11.2015}

    \section{Anordnung}
    Ein Körper $(K, +, \cdot)$ heißt \emph{angeordnet}, falls wir gewisse Elemente aus $K$ als \emph{positiv} auszeichen können. Schreibe $a \in K, a > 0$, falls gilt:
    \begin{enumerate}[i)]
        \item
            Es gilt für $a$ genau eine der drei Beziehungen:
            \begin{itemize}
                \item
                    $a > 0$
                \item
                    $a = 0$
                \item
                    $-a >0$
            \end{itemize}
        \item
            Für $a,b \in K, a > 0, b >0$ gilt: $a + b > 0$ und $a \cdot b > 0 $
    \end{enumerate}

    \section{Notation}
    Wir benutzen folgende Notation:
    \begin{itemize}
        \item
            $a > b :\Leftrightarrow a-b > 0$
        \item
            $a < b :\Leftrightarrow b > a$
        \item
            $a \geq b:\Leftrightarrow a > b \lor a = b$
        \item
            $a \leq b:\Leftrightarrow b \geq a$
        \item
            $max(a,b) :=
            \begin{cases}
                a \text{ falls } a > b \\
                b \text{ sonst }\\
            \end{cases}
            $
        \item
            $min(a,b) := 
            \begin{cases}
                a \text{ falls } a < b \\
                b \text{ sonst }\\
            \end{cases}
            $
    \end{itemize}

    \section{Beispiele}
    \begin{itemize}
        \item
            $\QQ$ und $\RR$ sind angeordnete Körper.
        \item
            $\CC$ ist kein angeordneter Körper.
        \item
            $\FF_p$ (zur Primzahl $p$) ist kein angeordneter Körper.
    \end{itemize}
    Insbesondere lassen sich $\CC$ und $\FF_p$ \emph{nicht} (durch besondere Tricks) anordnen!

    \section{Anordnungsgesetze}
    Sei $(K, +, \cdot)$ ein angeordneter Körper und $a,b,c \in K$, so gilt:
    \begin{enumerate}[i)]
        \item
            $a<b$, $b<c \Rightarrow a<c$
        \item
            $a<b \Rightarrow a+c < b+c$
        \item
            $a < b \land c > 0 \Rightarrow a \cdot c < b \cdot c$
        \item
            $a<b \Leftrightarrow -a > -b$
        \item
            $0 \leq a < b$ und $0 \leq c < d$\\
            $\Rightarrow a c < b d$
    \end{enumerate}

    \section{Einbettung}
    Sei $K$ ein angeordneter Körper. Dann können wir $\NN$ in $K$ einbetten.
    \begin{align*}
        n_k := \underbrace{1_K + \dots + 1_K}_{n-mal} \in K
    \end{align*}
    Für $n \in \NN$.

    \subsection{Notation}
    Wir werden im Folgenden intuitiv $n$ mit $n_k$ identifizieren.

    \section{Bernoulli'sche Ungleichung}
    Sei $n \in \NN$ und $-1 < x \in K$, so gilt:
    \begin{align*}
        1+ nx \leq (1 + x)^n
    \end{align*}

    \section{Betrag}
    Sei $K$ ein angeordneter Körper. Für $a \in K$ definieren wir:
    \begin{align*}
        |a| :=
        \begin{cases}
            a &\text{ falls } a \geq 0\\
            -a &\text{ sonst }
        \end{cases}
    \end{align*}
    Und nennen $|a|$ den (Absolut-)Betrag von $a$.

    \section{Eigenschaften Betrag}
    Es gilt für $a, b \in K$:
    \begin{enumerate}[i)]
        \item
            $|-a| = |a|$
        \item
            $|ab| = |a| |b|$
        \item\label{DU}
            $|a  + b| \leq |a| + |b|$ 
    \end{enumerate}
    Und nennen \ref{DU}) die \emph{Dreiecksunleichung}.

    \section{Archimedisch}
    Wir nennen einen angeordneten Körper \emph{archimedisch}, falls für alle $x \in K$ ein $n \in \NN$ existiert, so dass $n > x$ ist.

    \section{Folgerungen Archimedes}
    Ist $K$ ein archimedischer Körper, so folgt:
    \begin{enumerate}[i)]
        \item
            Zu jedem $\varepsilon > 0$ existiert ein $N \in \NN$, so dass $\frac{1}{n} < \varepsilon$ für alle $n \geq N$ gilt.
        \item
            Ist $b = 1 + \varepsilon$ mit $\varepsilon > 0$, so existiert für alle $ R \in K$ ein $N \in \NN$, so dass $b^n > R$ für alle $n \geq N$.
        \item
            Ist $0 < q < 1$, so gibt es ein $\varepsilon > 0$ und $n \in \NN$, so dass $q^n < \varepsilon$ für alle $n \geq N$.
    \end{enumerate}

    \section{Behauptung}
    $\RR$ ist ein angeordneter Körper

    \section{Weiterführung}
    Momentan gilt noch $\RR = \QQ$. Bald werden wir untersuchen, worin sich diese beiden Mengen unterscheiden, sprich $\RR \setminus \QQ$ betrachten.

    \lecturedate{04.11.2015}

    \chapter{Folgen in $\RR$}

    \section{Definition}
    Eine \emph{Folge} reeller Zahlen heißt ist eine Abbildung $\NN \to \RR$ mit $n \mapsto a_n$
    Wir schreiben die Folge als $(a_n)_{n \in \NN}$ oder $(a_1, a_2, a_3, \dots )$.

    \section{Beispiele}
    \begin{enumerate}[i)]
        \item
            $a_n = 2$ ist die Folge $(2, 2, 2, 2, \dots )$
        \item
            $a_n = n $ ist die Folge $(1, 2, 3, 4, 5, \dots)$
        \item
            $a_n = \frac{1}{n}$ ist die Folge $(\frac{1}{1},\frac{1}{2},\frac{1}{3}, \frac{1}{4}, \frac{1}{6}, \frac{1}{7}, \dots)$
    \end{enumerate}

    \section{Konvergenz}
    Eine Folge $(a_n)_{n \in \NN}$ heißt \emph{konvergent} mit Grenzwert $a \in \RR$, falls
    \begin{align*}
        \forall \varepsilon > 0 :\exists N \in \NN :\forall n \geq N: |a - a_n| < \varepsilon
    \end{align*}
    und schreiben $\lim\limits_{n \to \infty} a_n = a$.

    \section{Inverse Ungleichung}
    Für $x,y \in \RR$ gilt die \emph{inverse Dreiecksungleichung}:
    \begin{align*}
        ||x| - |y| | \leq |x - y|
    \end{align*}

    \section{Beispiele}
    \begin{enumerate}[i)]
        \item
            $a_n = 2$ ist eine konvergente Folge mit Grenzwert 2. 
        \item
            $a_n = n$ konvergiert nicht.
        \item
            $a_n = \frac{1}{n}$ konvergiert gegen 0. 
    \end{enumerate}

    \section{Eindeutigkeit}
    Der Grenzwert einer konvergenten Folge ist eindeutig bestimmt.

    \section{Sprechweise}
    Eine Folge die nicht konvergiert heißt \emph{divergent}.

    \section{Bemerkung}
    Eine konvergente Folge $(a_n)_{n \in \NN}$ konvergiert auch, wenn man endlich viele Folgenglieder weglässt. Insbesondere konvergiert $(a_n)_{n \geq n_0}$ mit $n_0 \in \NN$ gegen den selben Grenzwert.

    \section{Formel}
    Sei $x \in \RR$ mit $|x| < 1$ und $(s_n) = 1 + x + x^2 + \dots + x^n$. Dann konveriert $(s_n)$ mit
    \begin{align*}
        \lim\limits_{n \to \infty} (s_n) = \frac{1}{1-n}
    \end{align*}

    \section{Beispiele}
    Ausgelassen, da triviale Anwendungen obiger Formel.

    \section{Beschränktheit}
    Eine Folge $(a_n)_{n \in \NN}$ heißt \emph{nach oben (nach unten) beschränkt}, falls ein $k \in \RR$ existiert, so dass $a_n \leq k (a_n \geq k)$ für alle $n \in \NN$ ist. Eine Folge heißt beschränkt, wenn sie nach oben oder nach unten beschränkt ist.

    \section{Satz}
    Jede konvergente Folge ist beschränkt.

    \section{Einschachtelung}
    Für alle $a, \varepsilon \in \RR$ mit $\varepsilon > 0$ gilt:
    \begin{align*}
        a-\varepsilon < a < a + \varepsilon
    \end{align*}

    \section{Rechenregeln für Grenzwerte}\label{414}
    Seien $(a_n)$ und $(b_n)$ konvergente Folgen in $\RR$. Dann gilt:
    \begin{align*}
        \lim\limits_{n \to \infty} a_n \cdot \lim\limits_{n \to \infty} b_n &= \lim\limits_{n \to \infty} a_n \cdot b_n\\
        \lim\limits_{n \to \infty} a_n + \lim\limits_{n \to \infty} b_n &= \lim\limits_{n \to \infty} a_n + b_n\\
    \end{align*}

    \section{Lineartität}
    Das Bilden der Grenzwerte konvergenter Folgen ist linear, also ergänzend zu \ref{414} gilt:
    \begin{align*}
        \lim\limits_{n \to \infty} \lambda a_n = \lambda \lim\limits_{n \to \infty} a_n
    \end{align*}

    \lecturedate{09.11.2015}

    \section{Beispiel}
    Sei $a_n = \frac{n+1}{n} = \frac{n}{n} + \frac{1}{n} = 1 + \frac{1}{n}$. Dann gilt:
    \begin{align*}
        \lim\limits_{n \to \infty} a_n = 1 + 0 = 1
    \end{align*}

    \section{Quotienten von Folgen}
    Seien $(a_n)$ und $(b_n)$ konvergente Folgen mit $\lim\limits_{n \to \infty} b_n \neq 0$, dann gibt es ein $n_0 \in \NN$, für alle $n \in \NN_{\geq n_0}$ gilt:
    \begin{align*}
        \lim\limits_{n \to \infty} \frac{a_n}{b_n} = \frac{\lim\limits_{n \to \infty} a_n}{\lim\limits_{n \to \infty} b_n}
    \end{align*}

    \section{Rechenbeispiel}
    \begin{align*}
        a_n &= \frac{5n^2 + 2n + 1}{3n^2 + 10n}\\
        &= \frac{5 + \frac{2}{n} + \frac{1}{n^2}}{3 + \frac{10}{n}} \xrightarrow{n \to \infty} \frac{5}{3}
    \end{align*}

    \section{Ordnung von Grenzwerten}
    Seien $(a_n)$, $(b_n)$ und $(c_n)$ konvergente Folgen in $\RR$.
    \begin{enumerate}[i)]
        \item
            Ist $a_n \leq b_n$ $\forall n \in \NN$, so ist $\lim\limits_{n \to \infty} a_n \leq \lim\limits_{n \to \infty} b_n$
        \item\label{Sandwichlemma}
            Ist $\lim\limits_{n \to \infty} b_n = b = \lim\limits_{n \to \infty}$ und $b_n < a_n < c_n$, so konvergiert $(a_n)$ ebenfalls gegen $b$.
    \end{enumerate}
    \ref{Sandwichlemma} wird häufig auch als Sandwichlemma bezeichnet, da anschaulich $a_n$ von $b_n$ und $c_n$ eingeengt - \glqq gesandwicht\grqq\ - wird.

    \section{Ordnung im Grenzübergang}
    Sind $a_n < b_n$ konvergente Folgen in $\RR$ mit gleichem Grenzwert, so gilt für deren Grenzwerte im Allgemeinen nur
    \begin{align*}
        \lim\limits_{n \to \infty} a_n \leq \lim\limits_{n \to \infty} b_n
    \end{align*}

    \section{Beispiel}
    Der Grenzwert von $(\frac{n}{2^n})_{n \in \NN}$ ist 0.
    \begin{proof} Der Beweis erfolgt durch Anwendung des Sandwichlemmas auf die Nullfolge und $(q^n \cdot \lambda )$ \end{proof}

    \section{Bestimmte Divergenz}
    Eine divergente Folge $(a_n)_{n \in \NN}$ nennen wir \emph{bestimmt divergent} (oder \emph{uneigentlich konvergent}) gegen $\pm \infty$, wenn gilt:
    \begin{align*}
        \forall K \in \RR: \exists N \in \NN: \forall n \geq N: a_n > K \text{ (bzw $a_n < K$)}
    \end{align*}
    Wir notieren den Grenzwert als $\lim\limits_{n \to \infty} a_n = \pm \infty$.

    \section{Beispiele}
    \begin{enumerate}[i)]
        \item
            $a_n = n$ konvergiert uneigentlich gegen $\infty$.
        \item
            $a_n = -n^2$ konvergiert uneigentlich gegen $- \infty$.
        \item
            $a_n = (-1)^n$ divergiert.
    \end{enumerate}

    \section{Kehrwert von Grenzwerten}
    Sei $(a_n)$ eine relle Folge mit $a_n > 0$. Dann gilt:
    \begin{align*}
        \lim\limits_{n \to \infty} a_n = \infty \Leftrightarrow \lim\limits_{n \to \infty} \frac{1}{a_n} = 0
    \end{align*}

    \lecturedate{11.11.2015}    

    \section{Definition: Reihe}
    Sei $(a_n)_{n \in \NN}$ eine Folge, so nennen wir
    \begin{align*}
        (s_n)_{n \in \NN} := \sum_{k=1}^\infty a_k = \lim\limits_{n \to \infty} \sum\limits_{k=1}^n a_k
    \end{align*}
    eine \emph{unendliche Reihe} und notieren $\sum\limits_{k=1}^\infty a_k = s$, falls $s_n$ gegen $s$ konvergiert.

    \section{Beispiele}
    \begin{enumerate}[i)]
        \item
            \emph{Geometrische Reihe}: Ist $x \in \RR$ mit $|x| < 1$, dann ist $\sum\limits_{n=0}^\infty x^n = \frac{1}{1-x}$
        \item
            $\sum\limits_{n=1}^\infty \frac{1}{k \cdot (k+1)} = 1$
        \item
            $\sum\limits_{k=1}^\infty \frac{1}{k^2} = \frac{\pi^2}{6}$
        \item
            $\sum\limits_{k=1}^\infty \frac{1}{k} = \infty$
    \end{enumerate}

    \chapter{Vollständigkeitsaxiom}
    \section{Motivation}
    Wir wissen, dass $\QQ$ und $\RR$ angeordnete Körper sind. Aber offensichtlich unterscheiden sie sich noch in irgendwelchen Zahlen, welche wir noch nicht fassen können.

    \section{Cauchy-Folge}
    \begin{enumerate}[i)]

    \end{enumerate}
    Eine Folge $(a_n)_{n \in \NN}$ heißt \emph{Cauchy-Folge}, genau dann, wenn
    \begin{align*}
        \forall \varepsilon > 0 :\exists N \in \NN: \forall m,n \geq N: |a_n = a_m| < \varepsilon
    \end{align*}

    \section{Satz}
    Folgende Aussagen über die Folge $(a_n)_{n \in \NN}$ sind äquivalent:
    \begin{itemize}
        \item
            $(a_n)$ ist eine Cauchy-Folge.
        \item
            $(a_n)$ ist eine konvergente Folge.
    \end{itemize}

    \section{Vollständigkeitsaxiom für $\RR$}
    Jede Cauchy-Folge reeller Zahlen konvergiert gegen einen Grenzwert in $\RR$.

    \section{Bemerkung}
    \begin{enumerate}[i)]
        \item
            Das Vollständigkeitsaxiom ist unabhängig von den anderen (Körper-)Axiomen.
        \item
            Mit dem Cauchy-Kriterium lässt sich Konvergenz überprüfen, ohne den Grenzwert zu kennen.
    \end{enumerate}

    \section{Notation}
    Für $a,b \in \RR$ mit $a < b$ bezeichnen wir
    \begin{align*}
        [a,b] &:= \lbrace x \in \RR \mid a \leq x \leq b \rbrace \text{ als das \emph{abgeschlossene Intervall} von $a$ nach $b$}\\
        (a,b) := ]a,b[ &:= \lbrace x \in \RR \mid a < x < b \rbrace \text{ als das \emph{offene Intervall} von $a$ nach $b$}\\
        [a,b) := [a,b[ &:= \lbrace x \in \RR \mid a \leq x < b \rbrace \text{ als das \emph{halboffene Intervall} von $a$ nach $b$}
    \end{align*}

    \newcommand{\sebsut}{\supset}
    \newcommand{\sebsuteq}{\supseteq}

    \section{Intervallschachtelungsprinzip}
    Sei $I_1 \sebsut \dots \sebsut I_n \sebsut \dots$ eine Folge abgeschlossener Intervalle mit $\lim\limits_{n \to \infty} |I_n| = 0$, dann ist
    \begin{align*}
        \bigcap_{n \in \NN} I_n = \lbrace a \rbrace
    \end{align*}
    eindeutig.

    \section{Satz}
    Das Vollständigkeitsaxiom und Intervallschachtelungsprinziip sind äquivalent.

    

\end{document}