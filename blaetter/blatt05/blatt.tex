\documentclass{scrreprt}

\usepackage[utf8]{inputenc}
\usepackage{enumerate}
\usepackage{ngerman}
%\usepackage[german]{babel}

\usepackage{amsmath}
\usepackage{amsfonts}
\usepackage{amssymb}
\usepackage{amsthm}
\usepackage{mathtools}

\usepackage[hidelinks]{hyperref}

\makeatletter
\def\moverlay{\mathpalette\mov@rlay}
\def\mov@rlay#1#2{\leavevmode\vtop{%
   \baselineskip\z@skip \lineskiplimit-\maxdimen
   \ialign{\hfil$\m@th#1##$\hfil\cr#2\crcr}}}
\newcommand{\charfusion}[3][\mathord]{
    #1{\ifx#1\mathop\vphantom{#2}\fi
        \mathpalette\mov@rlay{#2\cr#3}
      }
    \ifx#1\mathop\expandafter\displaylimits\fi}
\makeatother

\newcommand{\NN}{\mathbb{N}}
\newcommand{\ZZ}{\mathbb{Z}}
\newcommand{\QQ}{\mathbb{Q}}
\newcommand{\RR}{\mathbb{R}}
\newcommand{\CC}{\mathbb{C}}
\newcommand{\cupdot}{\charfusion[\mathbin]{\cup}{\cdot}}
\newcommand{\bigcupdot}{\charfusion[\mathop]{\bigcup}{\cdot}}


\begin{document}

	\title{Analysis 1}
 	\author{Dominic Zimmer}
 	\subtitle{Übungsblatt 5, Abgabe}
 	\publishers{Übungsgruppe: Rami Ahmad}
  	\maketitle 

	%see http://www.namsu.de/Extra/strukturen/Inhaltsverzeichnis.html

	\section*{Aufgabe 1}
	Gegeben sind die Folge $(f_n)_{n \in \NN_{>0}}$ definiert durch $f_0 = 0$, $f_1= 1$ und $f_{n+1} = f_n + f_{n-1}$, die Folge $(a_n)_{n \in \NN} = \frac{f_{n+1}}{f_n}$ und der \emph{goldene Schnitt} $g$ als di eindeutig bestimmte, positive Lösung der Gleichung $g^2 = 1+g$.
	\begin{align*}
		 & g^2 & = & 1+g 	\\
		\Leftrightarrow & \left(g - \frac{1}{2} \right)^2 & = & \frac{5}{4}\\
		\Rightarrow & g & = & \frac{1 + \sqrt{5}}{2}\\
    \end{align*}
	\begin{enumerate}[(a)]
		\item
            Zu zeigen ist:
            \begin{align*}
                g = & 1 + \frac{1}{g}\\
                \Leftrightarrow g \cdot g = & 1 \cdot g + \frac{1 \cdot g}{g}\\
                \Leftrightarrow g^2 = & 1 + g~~~ \square
            \end{align*}
            und
            \begin{align*}
                a_{n+1} = & 1 + \frac{1}{a_n}\\
                \Leftrightarrow \frac{f_{n+2}}{f_{n+1}} = & 1 + \frac{f_n}{f_{n+1}}\\
                \Leftrightarrow f_{n+2} = & 1 \cdot f_{n+1} + \frac{f_n \cdot f_{n+1}}{f_{n+1}}\\
                \Leftrightarrow f_{n+2} = & f_{n+1} + f_n ~~~ \square
            \end{align*}
	\pagebreak
        \item
            Durch Induktion ist zu zeigen, dass
            \begin{align*}
                \vert a_n - g \vert = \frac{1}{f_n g^n}
            \end{align*}
            für alle $n \in \NN$ gilt.
            \begin{proof}
                Induktionsanfang: $n=1$
                \begin{align*}
                    \vert a_n - g \vert = & \vert a_1 - g \vert&\\ 
                    = & \left\vert \frac{f_2}{f_1} - g\right\vert&\\
                    = & \left\vert 1 - g \right\vert&\\
                    = & \left\vert 1 - (1 + \frac{1}{g}) \right\vert&\\
                    = & \left\vert - \frac{1}{g} \right\vert&\\
                    = & \frac{1}{g}&\\
                    = & \frac{1}{f_1 g} &= \frac{1}{f_n g} ~~~ \text{(ok)}\\
                \end{align*}
                Induktionsvorraussetzung: Sei die Behauptung bereits gezeigt für ein festes aber beliebiges $n \in \NN$.\\
                Induktionsschritt: $n \leadsto n+1$
                \begin{align*}
                    \left\vert a_{n+1} - g \right\vert =  & \left\vert 1 + \frac{1}{a_n} - 1 - \frac{1}{g} \right\vert ~~~ \text{nach (a)} &\\
                    = & \left\vert \frac{1}{a_{n}} - \frac{1}{g} \right\vert&\\
                    = & \left\vert \frac{g - a_n}{a_{n} \cdot g} \right\vert&\\
                    = & \frac{a_n - g}{a_n \cdot g}&\\
                    \overset{IV}{=} & \frac{\frac{1}{f_n \cdot g^n}}{a_n \cdot g}&\\
                    = & \frac{1}{f_n \cdot g^{n+1} \cdot a_n} & =  \frac{1}{f_{n+1} \cdot g^{n+1} }
                \end{align*}
            \end{proof}
        \item
        	Wir wollen zuerst zeigen, dass $(a_n)_{n \in \NN}$ tatsächlich gegen einen Wert $g \in \RR$ konvergiert. Sei $N \in \NN$ und $\varepsilon > 0 $ vorgegeben. So gilt
        	\begin{align*}
        		\left\vert a_n - g\right\vert \overset{(b)}{=} \frac{1}{f_n \cdot g^n} \leq \varepsilon ~~~ \forall n \geq N,
        	\end{align*}
        	da $g \approx 1.618 > 1$, womit $g^n$ für wachsende $n$ beliebig groß wird, und $f_n$ für alle $n \in \NN$ eine Summe von bereits positiven Zahlen darstellt, somit auch positiv ist und monoton wächst. Somit wächst für hinreichend große $n (>N)$ der Nenner als Produkt positiver Zahlen beliebig womit die Abschätzung für jedes vorgegebene $\varepsilon >0$ hält.\\
        	Nachdem die Konvergenz gezeigt ist, betrachten wir den Grenzwert. Im Grenzübergang konvergieren $a_n$ und $a_{n+1}$ gegen den gleichen Grenzwert, womit folgt:
        	\begin{align*}
        		& \lim\limits_{n \to \infty} a_{n+1} & = & \lim\limits_{n \to \infty} 1 + \frac{1}{a_n}\\
        		\Leftrightarrow & g & = & 1 + \frac{1}{g}\\
        	\end{align*}
        	Diese Gleichung kennen wir bereits aus Teilaufgabe (a) und wissen, dass sie vom \emph{goldenen Schnitt} erfüllt wird. Somit ist der Grenzwert der Goldene Schnitt $g = \frac{1 + \sqrt{5}}{2}$.
	\end{enumerate}

    \section*{Aufgabe 2}
    Zu zeigen: Sei $f \colon \RR \to \RR$ eine Abbildung, $0 \leq q < 1$, fest mit
    		\begin{align*}
    			\vert f(x) - f(y) \vert \leq q \vert x - y \vert ~~~ \forall x,y \in \RR
    		\end{align*}
    Dann gibt es genau einen \emph{Fixpunkt} $x \in \RR$ mit $f(x) = x$.
    \begin{proof}\\
    \begin{enumerate}[(a)]
    	\item
    		Wir zeigen die Eindeutigkeit der Fixpunkte. Nehmen wir an $x$ und $\hat{x}$ seien zwei verschiedene Fixpunkte, dann gilt:
    		\begin{align*}
    			x = f(x) ~~~ \text{ und } ~~~ \hat{x} = f(\hat{x})
    		\end{align*}
    		Wir betrachten den Abstand der beiden Fixpunkte:
    		\begin{align*}
    			\left\vert x - \hat{x} \right\vert = & \left\vert f(x) - f(\hat{x}) \right\vert\\
    			\leq & q \cdot \left\vert x - \hat{x}\right\vert
    		\end{align*}
    		Diese Ungleichung ist aber nur für $\vert x - \hat{x}\vert = 0$ erfüllbar. Somit muss folgen, dass $x = \hat{x}$ gilt. Dies ergibt allerdings einen Widerspruch zur Annahme, dass $x$ und $\hat{x}$ verschiedene Fixpunkte seien. Somit sind Fixpunkte eindeutig. 
    	\item
    		Sei die Folge $(x_n)_{n \in \NN}$ rekursiv definiert durch $x_n = f(x_{n_1})$. Für alle $n \in \NN$ gilt:
    		\begin{align*}
    			\vert x_{n+1} - x_n \vert \leq q^n \vert x_1 - x_0 \vert
    		\end{align*}
    		Wir zeigen die Behauptung durch Induktion:\\
    		Induktionsanfang: $n = 1$
    		\begin{align*}
    			\vert x_{2} - x_1 \vert = & \vert f(x_1) - f(x_0) \vert\\
    			\leq & q \cdot \vert x_1 - x_0 \vert ~~~ (ok)
    		\end{align*}
    		Induktionsvorraussetzung: Sei die Behauptung bereits gezeigt für ein festes aber beliebiges $n \in \NN$.\\
            Induktionsschritt: $n \leadsto n+1$
            \begin{align*}
            	\vert x_{n+1} - x_n \vert = &\vert f(x_{n} - f(x_{n-1}) \vert\\
            	\leq & q \cdot \vert x_n - x_{n-1} \vert \\
            	\overset{IV}{=} & q \cdot q^n \vert x_1 - x_0\vert \\
            	= q^{n+1} \cdot \vert x_1 - x_0 \vert
            \end{align*}
    	\item
    		Sei $x \in \RR$ der Grenzwert der Folge $(x_n)_{n \in \NN}$, $\varepsilon >0$ vorgegeben und $n \leq m \in$ Indizes. Wir betrachten:
    		\begin{align*}
    			\vert x_n - x_m \vert \leq & \sum\limits_{i = n+1}^m \vert x_i - x_{i-1}\vert\\
    			\leq & \left( \sum\limits_{i = n+1}^m q^{i-1}\right) \vert x_1 - x_0\vert\\
    			\leq & \left( q^n \sum\limits_{i=0}^\infty q^i\right) \vert x_1 - x_0 \vert\\
    			= & \underbrace{\frac{q^n}{1-q}}_{\text{:= $a$}} \underbrace{\vert x_1 - x_0 \vert}_{\text{beschränkt}} \leq \varepsilon
    		\end{align*}
    		Da $0 \leq q < 1$, wird $a$ beliebig klein und somit ist $x_n$ eine Cauchy Folge.
    		Wir betrachten nun die rekursive Definition von $(x_n)$ und sehen, dass die Folge
    		\begin{align*}
    			&\lim\limits_{n \to \infty} x_n & = & \lim\limits_{n \to \infty} f(x_{n-1})\\
    			\Leftrightarrow & x &=& f(x)
    		\end{align*}
    		gegen den Fixpunkt $x$ konvergiert.
    \end{enumerate}
    \end{proof}

    \section*{Aufgabe 3}
    Die Folgen $(a_{2n})_{n \in \NN}$, $(a_{2n + 1})_{n \in \NN}$ und $(a_{5n})_{n \in \NN}$ konvergieren. Konvegriert auch $(a_n)_{n \in \NN}$?
    \begin{proof}
    Wir betrachten $(a_{5n})_{n \in \NN}$. Diese Teilfolge besteht aus den Gliedern ($a_5, a_{10}, a_{15}, \dots)$. Somit teilt sie sich unendlich viele Folgenglieder mit $(a_{2n})_{n \in \NN}$ und mit $(a_{2n +1})_{n \in \NN}$, nämlich $(a_{5}, a_{15}, a_{25}, \dots)$ bzw. $(a_{10}, a_{20}, a_{30}, \dots)$. Da wir von $(a_{5n})_{n \in \NN}$ bereits wissen, dass die Folge konvergiert, konvergieren auch ihre Teilfolgen - insbesondere gegen den gleichen Grenzwert.
    Somit konvergieren $(a_{2n})$ und $(a_{2n + 1})$ gegen den gleichen Grenzwert wie $(a_{5n})$.
    Da sich $(a_n)_{n \in \NN}$ disjunkt in die zwei Teilfolgen $(a_{2n})$ und $(a_{2n + 1})$ zerlegen lässt, konvergiert die Folge $(a_{n})$ ebenfalls gegen den gleichen Grenzwert.
    \end{proof}

    \section*{Aufgabe 4}
    Jede Folge $(a_{n})$ reeller Zahlen besitzt entweder eine konvergente oder eine bestimmt divergente Teilfolge.
    \begin{proof}
    Wir betrachten zwei Fälle:
    \begin{itemize}
        \item
            Die Folge sei beschränkt.\\
            Nach dem Satz von \emph{Bolzano-Weierstraß} hat $(a_n)_{n \in \NN}$ eine konvergente Teilfolge. Somit gilt die Aussage.
        \item
            Die Folge ist unbeschränkt.\\
            Sei ohne Einschränkung die Folge nach oben unbeschränkt (der Beweis für "`nach unten unbeschränkt"' verläuft ganz analog). Wir konstruieren eine Folge von Indizes $(i_n) = i_1, i_2, \dots$ unter der Invariante, dass $a_{i_k} < a_{i_{k+1}}$. Dass eine solche Folge existiert, folgt daraus, dass die Folge $(a_n)_{n \in \NN}$ unbeschränkt ist. Betrachten wir nun die Teilfolge $(a_{i_n})_{n \in \NN}$. Nach Konstruktion gilt für den Grenzwert
            \begin{align*}
                \lim\limits_{n \to \infty} (a_{i_k}) = \infty
            \end{align*}
    \end{itemize}
    \end{proof}

\end{document}